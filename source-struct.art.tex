\section{MaxWit g-bios简介}
%\section{g-bios: An Open Source Bootloader Project}
%MaxWit开放实验室(MaxWit Open Lab)是由多家公司资助成立的,致力于研发开源项目和探讨软件开发技术的公益性组织。2008年1月正式成立于上海浦东张江高科,目前开放实验室成员主要来源于Google、Intel、 TI、AMD、华为、Cisco、飞利浦等公司资深研发人员以及清华、浙大、上交大、中科院等科研院校的师生。

g-bios(以下简称g-bios)是由Intel、IBM、Qualcomm、AMD等公司的几名软件工程师与开源社区共同研发的一个Bootloader\footnote{或者说是一个嵌入式系统的BIOS,相当于PC机的BIOS+Bootloader。}。g-bios不但借鉴了几乎所有主流BSP/BIOS/Bootloader的优点,而且加入不少独创的特性,包括:
\begin{enumerate}\setlength{\itemsep}{-\itemsep}
\item 自动检测有待烧录的image文件类型,并智能自动烧录。
\item 支持多种文件系统,包括YAFFS2、JFFS2、CRAMFS、UBI、NFS等。
%\item 支持两种用户界面:GUI(类似传统PC BIOS)和命令行模式(面向嵌入式系统)。
\item 命令行自动补全(Tab)键及历史记录(上、下键)支持。
\item Flash(MTD)分区支持,帮助Linux、Android内核识别分区。
\item 自动设置Linux内核启动参数(Linux kernel command line),极大地降低了参数设置的复杂度并减少了启动出错的概率。当然,同时也支持手动设置,以满足特殊要求。
\item 常用命令具有记忆功能。如boot命令,它能记住用户输入的参数,以后只需简单输入boot即可。
\item 引入全新的架构及NB技术(即Never Burn-down,又称``烧不死''技术)。
%核心设计思想是:把g-bios分为上半部分和下半部分,上半部分以最小的代码量完成CPU和Memory的初始化,并实现引导下半部分的功能;下半部分为g-bios主体。上半部分设计简单,调试周期短,完成后就固化在特定的引导区中不再更改;
开发人员可在没有仿真器的情况下大胆开发
Bootloader
%下半部分代码(即g-bios%主体)
。事实上,只需一根串口数据线应能轻松完成整个g-bios的开发。启动代码的地址无关性带来的麻烦?没有了!因为bug或不小心改错了代码,甚至是数据线连接问题而导致启动黑屏?也不可能出现了!
%在调试完成并正试发布的产品时,若有必要,也可将上下两部分可合成一个整体——只需一个命令重新编译即可。
\item 支持完整的中断机制。开发者可简单地通过一个编译选项选择IRQ或Polling两种模式的中的任意一种。
\item 优秀的网络子系统,并提供符合POSIX规范的Socket API,方便二次开发。
%\item 优秀的软件架构及子系统设计,包括:中断、网络、Flash、USB子系统,等等。
%\item 集成类似PC机版本的Video BIOS。
%\item 支持基于龙芯的PC机及嵌入式系统。
\item 支持
%嵌入式系统几乎所有
多种常用外设,包包括:WDT、UART、NAND、NOR、SD/MMC、USB、LCD、Touchscreen,...
\item 集成硬件调试/测试程序,大大提高了bring-up的工作效率。
\item 完美支持Google Android操作系统,简化Android的系统移植过程。
\item 支持图形化配置,不但让新手很容易上手,而且使g-bios的移植和开发过程变得更简单。
\end{enumerate}

更多详情,请登录项目主页http://maxwit.googlecode.com或ChinaUnix论坛上的g-bios版块(http://bbs.chinaunix.net/forum-238-1.html)。

\section{获取源码}
请确认git(一个版本管理软件)已经安装,然后执行如下命令:
\begin{lstlisting}[language=bash,numbers=none]
$ git clone git://github.com/maxwit/g-bios.git
\end{lstlisting}
此时会在当前目录(方便描述起见,假定为HOME目录)下将会创建一个名为 ``g-bios''的目录,该目录中为g-bios源码。
%\section{如何参与g-bios开发}
%g-bios开源社区采用maillist和bbs相结合的方式,任何人都可以通过这两种方式把自己的代码递交给g-bios项目维护者。若对文档有任何疑问或改进也可联系我们。
%  \begin{table}[htbp]
%  \centering
%  \setlength{\parindent}{0pt}
%  \begin{tabular}{|c|c|}
%  \hline
%  g-bios论坛 &\small http://linux.chinaunix.net/bbs/forum-70-1.htm \\
%  \hline
%  g-bios邮件列表 &\small maxwit@googlegroups.com \\
%  \hline
%  g-bios项目维护者 &\smallConke Hu $<$conke.hu@gmail.com$>$ \\
%					 &\smallTiger Yu $<$tigerflying.yu@gmail.com$>$ \\
%					 &\smallFleya Hou $<$fleya.hou@gmail.com$>$ \\
%  \hline
%  文档编辑 &\small  \\
%  \hline
% \end{tabular}
% \end{table}

\section{g-bios体系结构}

\section{源码目录结构}
\begin{lstlisting}[language=bash, numbers=none]
$ ls /g-bios/
bh  build  doc  include  Makefile  th
\end{lstlisting}

\begin{equation*}
\text{g-bios}
\left\{
	\begin{aligned}
	&\text{th: stage 1代码} \\
	&\text{bh: stage 2代码,即g-bios上半部分启动代码} \\
	&\text{Makefile:} 整个g-bios项目的顶层Makefile \\
	&\text{include:头文件(stdio.h, string.h, g-bios.h等)} \\
	&\text{doc:使用开发手册} \\
	&\text{build:与编译相关文件夹(配置文件,编译规则)} \\
	\end{aligned}
\right.
\end{equation*}


bh目录:
\begin{lstlisting}[language=bash, numbers=none]
$ ls /g-bios/bh
app  arm  base  driver  filesys  lib  Makefile  mm
\end{lstlisting}

\begin{equation*}
\text{bh: stage 2}
\left\{
	\begin{aligned}
	&\text{app:g-bios应用程序、命令所在目录(ping, ls, reboot等)} \\
	&\text{arm:arm体系结构相关代码} \\
	&\text{base:g-bios下半部分公共文件夹(main, shell, sysconfig等)} \\
	&\text{driver: 驱动程序源码,通用driver} \\
	&\text{filesys:文件系统相关源码} \\
	&\text{lib:库文件(stdlib, string, net等)} \\
	&\text{mm:内存管理文件夹(内存分配函数代码等)}
	\end{aligned}
\right.
\end{equation*}

app目录

\begin{lstlisting}[language=bash, numbers=none]
$ ls /g-bios/app
boot  flash  led  Makefile  memory  misc  net  serial  shell  sysconf
\end{lstlisting}
将不同功能的application分成类,置于不同的文件夹下。每个文件下都有一个Makefile文件。从文件夹的命名就可知道其文件夹下有哪些文件。app文件下的每个.c文件都对应g-bios中的某个命令,文件名就是g-bios shell中的命令名。
\begin{equation*}
\text{bh/app}
\left\{
	\begin{aligned}
	&\text{boot:引导程序(boot kernel or reboot, etc)} \\
	&\text{flash:flash操作相关app} \\
	&\text{led:控制led灯程序} \\
	&\text{Makefile} \\
	&\text{memory:memory操作函数} \\
	&\text{misc:杂项程序} \\
	&\text{net:网络程序} \\
	&\text{serial:串口程序} \\
	&\text{shell: clear, ls, etc} \\
	&\text{sysconf: configure程序}
	\end{aligned}
\right.
\end{equation*}

目前g-bios仅支持arm体系结构的处理器,因此仅有arm目录。
\begin{lstlisting}[language={SH},numbers=none]
$ ls /g-bios/bh/arm/
arm_heap.c  at91sam926x  exception.c  g-bios-bh.lds  head.S  lib  Makefile  omap3530  s3c24x0  s3c6410  udelay.S
\end{lstlisting}

\begin{equation*}
\text{bh/arm}
\left\{
	\begin{aligned}
	&\text{arm\_heap.c:heap init} \\
	&\text{at91sam926x} \\
	&\text{exception.c:exception handle} \\
	&\text{g-bios-bh.lds:编译bh所需的链接脚本。} \\
	&\text{head.S:stage 2入口} \\
	&\text{lib:arm体系结构相关库文件} \\
	&\text{Makefile}\\
	&\text{omap3530}\\
	&\text{s3c24x0}\\
	&\text{s3c6410}\\
	&\text{udelay.S:udelay}
	\end{aligned}
\right.
\end{equation*}
其中at91sam926x、s3c24x0、s3c6410为平台相关代码:包括soc相有关的driver,board级初始化代码。

嵌入式开发要了解的几个概念:板级,SOC级,CPU级,CPU core级。板级:与硬件开发板的布线有关。现在嵌入式处理器,通常将SOC级,CPU级,CPU core级做成一颗芯片。也就是所说的SOC,例如s3c6410,s3c2440,at91sam9261等这就是一个SOC。每个芯片公司将一个常用的外设芯片集成在一个cpu芯片里,就生产出了自己的SOC处理器芯片。这也就是说不同的SOC,他们所集成的外设可能会不一样。因此对于不同的SOC,他们的相对的处理代码也就可能不一样了。SOC里集成了外设,所以也可将SOC叫做platform或``板子''。cpu core也就是所说的体系结构例如:arm v5,arm v6,arm v7,core 2等。cpu就将cpu core,MMU,cache等等集成在一起的芯片,不同的CPU,它们的cache个数、大小可能不同,因此处理cpu级的程序代码也可能不同。(例如:arm 9,arm 11,cotex-a8等)。其它厂商可以在此基础上加入一此外设从而制造处一个SOC。因此,对于不同的SOC,他们的cpu和cpu core可能相同(s3c2410,s3c2440他们的都是arm 9系列cpu,因此他们的cpu和cpu-core就是一样)。

有了以上基础后再来看ARCH相关的目录结构就十分简单了。s3c6410,at91sam926x,s3c24x0为不同cpu的处理程序,里含各个SOC处理的相关的代码。*.c,*.s体系结构公共相关的代码。lib体系结构相关库文件。

th目录:
\begin{lstlisting}[language={SH}]
$ ls /g-bios/th/
arm  base Makefile
\end{lstlisting}
g-bios上半部分启动程序,arm目录与上面所讲的arm目录类似, 里不再重述。
base目录上半部分与体系结构无关代码(main, ymodem, kermit等)。

driver目录:
\begin{lstlisting}[language=bash, numbers=none]
$ ls /g-bios/driver
flash  gpu  Makefile  mmc  net  uart
\end{lstlisting}
不同外设的驱动程序源码。

\begin{equation*}
\text{driver}
\left\{
	\begin{aligned}
	&\text{flash:flash驱动程序(nand flash, nor flash, dataflash等)}\\
	&\text{gpu:显卡驱动程序(lcd等)}\\
	&\text{mmc:mmc相关驱动(sd卡等)}\\
	&\text{net:net相关驱动(dm9000网卡驱动,cs8900驱动,协议栈等)}\\
	&\text{uart:串口相关驱动。}
	\end{aligned}
\right.
\end{equation*}
